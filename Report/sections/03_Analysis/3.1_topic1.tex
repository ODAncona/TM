\subsection{NixConfig vs NixFlakes}

\subsection{Base Image: NixOS vs Ubuntu}

Avantages
- Reproductibilité totale :
- Avec NixOS, tout (paquets, configuration système, users, SSH, firewall, services…) est versionné et reconstruit à l'identique sur n'importe quelle machine.
- Pas de « drift » entre les environnements : pas d'état caché, pas de scripts d'init qui peuvent diverger.
- Déclarativité :
- La configuration (modules Nix) décrit l'état final souhaité, pas une suite d'actions impératives.
- Flexibilité :
- Ajout/suppression de services, changement de version, ajout de paquets… tout est modulaire et paramétrable.
- Sécurité :
- Moins de dépendance à l'état antérieur de la VM : chaque build part de zéro.
- Facilité d'automatisation :
- Génération d'images (qcow2, ISO, etc.) automatisée, sans intervention manuelle ni scripts shell fragiles.

Limites de l'approche Ubuntu + cloud-init + Ansible
- Imprévisibilité :
- Les images Ubuntu officielles sont parfois modifiées, certains paquets ou configurations changent, ce qui peut casser des scripts.
- Scripts impératifs :
- Ansible applique des actions, mais l'état final dépend de l'ordre d'exécution, de l'état de l'image, etc.
- Tests/rebuilds plus difficiles :
- Pour tester un changement, il faut souvent relancer tout le pipeline (Terraform + Ansible), et le résultat n'est pas toujours identique.


\subsection{LibVirt vs Tofu}

Avantages
- Contrôle granulaire et dynamique :
- Le wrapper Python permet de piloter la création/destruction de VMs à la volée, d'intégrer de la logique métier, de la validation, etc.
- Intégration directe avec la configuration Python :
- Le modèle de cluster (Pydantic) est directement utilisé pour générer les VMs, sans traduction dans un autre langage (HCL).
- Extensibilité :
- Possibilité d'ajouter des hooks, des opérations post-création, de l'orchestration complexe, etc., plus difficile à faire avec Terraform.
- Pas de dépendance à un backend cloud :
- Libvirt est natif, léger, et adapté au lab/local/dev : pas besoin d'API cloud, ni de provider Terraform parfois instable.

Limites de Terraform
- Moins flexible pour l'orchestration dynamique :
- Terraform est fait pour décrire un état d'infrastructure, pas pour orchestrer des workflows dynamiques ou répondre à des événements.
- Double maintenance (HCL + Ansible + Python) :
- Nécessite de maintenir les fichiers HCL (Terraform), les playbooks Ansible, et éventuellement du code Python pour la logique métier.

\subsection{Python vs Ansible}

Avantages
- Typage fort et validation :
- Les modèles Pydantic valident la configuration dès le chargement : erreurs détectées avant même le provisionnement.
- Unification de la logique :
- Toute la logique de configuration, de validation, d'orchestration est dans le même langage, facilement testable.
- Réutilisabilité et modularité :
- Possibilité de générer dynamiquement des configurations, de les adapter, de les versionner, etc.

Limites d'Ansible
- Impératif et fragile :
- Les playbooks sont séquentiels, parfois non idempotents, et la validation du résultat est plus difficile.
- Multiplication des outils :
- Nécessite de maintenir des rôles, des inventaires, des variables, etc., souvent dispersés.

\subsection{StaticIP vs DHCP}

\subsection{Resource manager vs Schedulers}